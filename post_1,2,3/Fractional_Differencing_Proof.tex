\documentclass{article}
\usepackage{graphicx,amsfonts,amssymb,amsmath,latexsym,amsthm}
\setlength{\topmargin}{0in} \setlength{\textheight}{8.25in}
\setlength{\parindent}{0em}
\topmargin=-0.45in
\evensidemargin=0in
\oddsidemargin=0in
\textwidth=6.5in
\textheight=9.0in
\headsep=0.25in

\linespread{1.1}

\begin{document}

\begin{center}
{\Large Derivation of Fractional Differencing} \\
\medskip
\bigskip
{\Large Dhiraj Amarnani}
\end{center}

\vspace{0.4in}

{\bf Lemma 1}: Binomial Series is the Maclaurin Series for $f(x) = (1 + x)^d$ 

\bigskip

Explicitly:\\

\begin{equation} 
(1 + x)^d = \sum_{k=0}^{\infty} {d \choose k} x^k = 1 + dx + \frac{d(d-1)}{2!}x^2 + \frac{d(d-1)(d-2)}{3!}x^3 + ... = \sum_{k=0}^{\infty} \frac{\prod_{i=0}^{k-1} (d-i)}{k!} x^k
\end{equation}
 
 \bigskip
 
 {\bf Lemma 2}: Backshift Operator $B$ has the following Properties:
 
 \bigskip
 
 Given a matrix of time series values $X = {X_1, X_2, ...}$
 \bigskip

\begin{equation}
\begin{split}
B*B & = B^2, B + B = 2B\\ 
B^kX_t & = X_{t-k} \thinspace \mbox{for}\thinspace t > k\thinspace, \mbox{and for all integers}\thinspace, k > 0\\
B^{-1}X_t & = X_{t+k}\\
\end{split}
\end{equation}

\bigskip

For example: $(1-B)^2 = 1- 2B + B^2$ and $(1-B)^2 X_t = X_t -2X_{t-1} + X_{t-2}$

\bigskip
	
{\bf Derivation}: Show that $(1-B)^d$, the formula used to derive weights for fractional differencing, converges to an infinite series of weights $w = \{w_0, w_1, w_2, w_3, ...\}$

\bigskip

\begin{equation}
\begin{split}
(1-B)^d = \sum_{k=0}^{\infty} {d \choose k} (-B)^k  & = \sum_{k=0}^{\infty} \frac{\prod_{i=0}^{k-1} (d-i)}{k!}(-B)^k \\
& = \sum_{k=0}^{\infty} (-B)^k \frac{\prod_{i=0}^{k-1} (d-i)}{k!}\\
& = \sum_{k=0}^{\infty} (-B)^k \prod_{i=0}^{k-1} \frac{d-i}{k-i}\\
& = 1 - dB + \frac{d(d-1)}{2!}B^2 - \frac{d(d-1)(d-2)}{3!}B^3 + ... \\
\end{split}
\end{equation}

\bigskip

Also, note an interesting property below with the weighting scheme derived above:

$$w_k  = -w_{k-1} \frac{d-k+1}{k}$$

\medskip

\mbox{Consider the following:}\
$w_3   = -w_2 \frac{d-3+1}{3} = -(\frac{d(d-1)}{2!})(\frac{d-2}{3}) = -\frac{d(d-1)(d-2)}{3!}$


\end{document}
